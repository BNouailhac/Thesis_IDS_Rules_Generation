
Fichier de configuration Suricata utilisé lors de test de fichier. Les fonctionnalités non exploitées sont désactivées ("enabled: no") et les autres fonctionnalités (non montrées ici) sont laissées avec leurs valeurs par défaut.

\vspace{1em}

\begin{figure}[h]%
    \center%
\begin{lstlisting}[language=Bash]
%YAML 1.1
---
# This configuration file generated by Suricata 7.0.3.
suricata-version: "7.0"
# The default logging directory.
default-log-dir: /var/log/suricata/
# Global stats configuration
stats:
  enabled: yes
  # The interval field (in seconds) controls the interval at
  # which stats are updated in the log.
  interval: 8
# Configure the type of alert (and other) logging you would like.
outputs:
  # a line based alerts log similar to Snort's fast.log
  - fast:
      enabled: no
      filename: fast.log
  # Extensible Event Format (nicknamed EVE) event log in JSON format
  - eve-log:
      enabled: no
      filename: eve.json
      xff:
        enabled: no
        mode: extra-data
        header: X-Forwarded-For
  # output module to store certificates chain to disk
  - tls-store:
      enabled: no
  # By default all packets are logged except:
  # - TCP streams beyond stream.reassembly.depth
  # - encrypted streams after the key exchange
  - pcap-log:
      enabled: no
\end{lstlisting}
\end{figure}

\newpage

\begin{figure}[h]%
    \center%
\begin{lstlisting}[language=Bash]
  # a full alert log containing much information for signature writers
  # or for investigating suspected false positives.
  - alert-debug:
      enabled: no
      filename: alert-debug.log
      append: yes
  # Stats.log contains data from various counters of the Suricata engine.
  - stats:
      enabled: no
      filename: stats.log
      append: yes
  # a line based alerts log similar to fast.log into syslog
  - syslog:
      enabled: no
      # reported identity to syslog. If omitted the program name (usually
      # suricata) will be used.
      facility: local5
# Logging configuration. This is not about logging IDS alerts/events, but
# output about what Suricata is doing, like startup messages, errors, etc.
logging:
  # The default log level:
  default-log-level: Notice
  # Define your logging outputs.
  outputs:
  - console:
      enabled: yes
      level: Error
  - file:
      enabled: yes
      level: Error
      filename: suricata.log
  - syslog:
      enabled: yes
      facility: local5
      format: "[%i] <%d> -- "
rule-files:
  - suricata.rules

...
\end{lstlisting}
\end{figure}