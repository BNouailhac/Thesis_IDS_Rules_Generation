\begin{figure}[h]%
    \center%
\begin{lstlisting}[language=Python].
# Telechargement des events depuis le "feeds" mis en parametre dans un dossier local
def download_files(feed_url):
    if not os.path.exists(directory):
        os.makedirs(directory)
    os.system(f"wget -r -np -nd -A json -P ./events/ {feed_url}")
\end{lstlisting}
\end{figure}

\begin{figure}[h]%
    \center%
\begin{lstlisting}[language=Python].
# Reprend tous les evenements telecharges precedemment et les integre dans l'instance locale de MISP.
def import_events():
    events_files = glob.glob("./events/*.json")
    events = []
    for file in events_files:
        if (file != "manifest.json"):
            with open(file, 'r') as f:
                file = json.load(f)
            try:
                misp.add_event(file)
                print(f'Successfully imported to misp {file}')
            except Exception as e:
                print(f'Error importing {file}: {e}')
    return events
\end{lstlisting}
\end{figure}

\newpage

\begin{figure}[h]%
    \center%
\begin{lstlisting}[language=Python].
# Creer un evenement Pouvant etre exporter en regles suricata
def create_event(_distribution, _threat_level, _analysis, _info, _attribute, _tags):
    event = MISPEvent()

    # Votre choix de distribution:
    # 0 - This Organization only
    # 1 - This community only
    # 2 - Connected communities
    # 3 - All communities
    event.distribution = _distribution

    # niveau de menace:
    # 1 - High
    # 2 - Medium
    # 3 - Low
    # 4 - Undefined
    event.threat_level_id = _threat_level

    # Represente le niveau de maturite de l'analyse:
    # 0 - Initial
    # 1 - Ongoing
    # 2 - Complete
    event.analysis = _analysis
    event.info = _info
    event.date = datetime.now().date().isoformat() # Today's date

    # Ajout d'un attribut (IOC) que Suricata peut utiliser, tel qu'un domaine
    attr = MISPAttribute()
    attr.category = _attribute["category"]
    attr.type = _attribute["type"]
    attr.value = _attribute["value"]
    attr.to_ids = True  # Mark as to_ids to be used for IDS export
    attr.comment = _attribute["comment"]

    event.add_attribute(**attr)
    # Ajout de "tags"
    for tag in _tags:
        event.add_tag(tag)

    # Telecharge l'evenement sur l'instance local de MISP
    response = misp.add_event(event)
\end{lstlisting}
\end{figure}