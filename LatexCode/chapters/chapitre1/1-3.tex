%%%%%%%%%%%%%%%%%%%%%%%%%%%%
% SECTION                  %
%%%%%%%%%%%%%%%%%%%%%%%%%%%%
\vspace{1em}

Le stage s'est déroulé du 1er avril au 30 septembre 2024. Pendant cette période, j'ai bénéficié d'un bureau au sein du SOC-MC avec un ordinateur personnel connecté à un réseau séparé sur lequel j'ai été libre de réaliser tous les développements nécessaires à la réalisation de mon projet. Cette configuration avait l'avantage de me permettre d'être proche des personnes impliquées dans l'outil sur lequel je travaillais, ainsi tout au long de mon stage j'ai pu disposer de l'expérience et des conseils, d'une part, des analystes SOC de l'AMSN et, d'autre part, de mon tuteur de stage Bruno VALENTIN, qui m'a partagé les connaissances et les avis du CSIRT.\\

Le stage s’est déroulé en trois parties distinctes :
\vspace{0.5em}
\begin{enumerate}[itemsep=0.5em]
    \item Apprentissage des technologies et du contexte
    \item Développement de la solution
    \item Rédaction de la thèse\\
\end{enumerate}

La première partie correspond à mon premier mois de stage, au cours duquel j'ai pris connaissance de mon environnement de travail et des missions de l'Agence, ainsi que des différents corps qui la composent. Au cours des semaines inaugurales de mon stage, j'ai dédié le plus clair de mon temps à suivre le travail quotidien de mes collègues pour appréhender les outils qu'ils utilisaient et comprendre comment ils fonctionnaient.\\

Après cette phase d'apprentissage, j'ai pu commencer à développer la solution relative à 
la problématique visée. Cette phase a duré quatre mois, à l'issue desquels j'ai soumis la solution et la documentation à mon tuteur. Mon tuteur s'est chargé quant à lui de la mise en production de mon travail pendant le mois qui a suivi.\\

Le dernier mois de stage et jusqu'à son terme, ayant finalisé la mission qui m'avait été confiée par mon tuteur, j'ai pu me concentrer avec son appui à parfaire, au sein de l'agence, la rédaction de la présente thèse, en l'étoffant par des échanges constructifs.\\

Tout au long de ces périodes, Bruno VALENTIN et Sébastien ABBONDANZA ont suivi mon travail en programmant des rapports bimensuels au cours desquels j'ai régulièrement fait état de mes progrès et reçu des conseils de leur part.