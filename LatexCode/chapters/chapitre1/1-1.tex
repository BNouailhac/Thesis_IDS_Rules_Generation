%%%%%%%%%%%%%%%%%%%%%%%%%%%%
% SECTION                  %
%%%%%%%%%%%%%%%%%%%%%%%%%%%%

L’Agence Monégasque de Sécurité Numérique (AMSN)\footnote{Site officiel de l’Agence Monégasque de Sécurité Numérique : \url{https://amsn.gouv.mc/}}, créée par Ordonnance Souveraine le 23 décembre 2015\footnote{\href{https://journaldemonaco.gouv.mc/Journaux/2015/Journal-8257/Ordonnance-Souveraine-n-5.664-du-23-decembre-2015-creant-l-Agence-Monegasque-de-Securite-Numerique}{Ordonnance Souveraine n° 5.664 du 23 décembre 2015}}, est l’autorité nationale en charge de la sécurité des systèmes d’information. Fondée sur le modèle de l'Agence Nationale de la Sécurité des Systèmes d'Information (ANSSI)\footnote{Site officiel du CERT-FR : \url{https://www.cert.ssi.gouv.fr/}} française, l'Agence est sous l’autorité directe du Ministre d’État (équivalent du Premier ministre français) et a pour rôle de constituer un centre d’expertise, de réponse et de traitement en matière de sécurité et d’attaques numériques pour l’Etat et les Opérateurs d’Importance Vitale (OIV) monégasques.\\

\vspace{1em}

L'AMSN est aujourd'hui structurée autour de deux pôles, exposés ci-après.

\newpage

\subsection{Le Pôle Expertise }

\vspace{1em}

Le Pôle Expertise forme un groupe d'experts qui joue un rôle crucial dans la conception, la mise en œuvre et le suivi des stratégies de cybersécurité de l'État, ainsi que dans la sensibilisation et l'évaluation de la sécurité de l'infrastructure numérique nationale. Ses missions peuvent être résumées comme suit :\\

\begin{itemize}[itemsep=1em]
    \item[•] Conseiller et coordonner les travaux interministériels sur la sécurité des systèmes d'information.
    \item[•] Contrôler l'application des mesures de sécurité adoptées par le Gouvernement sur les systèmes d'information des administrations et des opérateurs publics ou privés.
    \item[•] Sensibiliser les services publics et les opérateurs publics et privés aux exigences en matière de sécurité numérique.
    \item[•] Évaluer et qualifier les acteurs et services de sécurité numérique de la Principauté.
    \item[•] Mise en place et maintenance du service national de certification électronique pour les services de l'État, de la Commune, ainsi que les personnes physiques ou morales autorisées.
\end{itemize}

\vspace{1em}

Dans le cadre de mon stage, il m'a parfois été donné l'occasion de côtoyer le Pôle Expertise. Ce qui m'a permis d'assister à certaines de leurs activités quotidiennes ainsi que de bénéficier de leur point de vue sur l'orientation de ma thèse.

\newpage

\subsection{Le Centre de réponse et de traitement en matière\\ d’attaques numériques (CERT-MC) }

\vspace{1em}

Afin d'être à même  de participer à la coopération internationale face aux menaces numériques, l'AMSN s'est dotée d'un CERT suivant les standards établis par le FIRST\footnote{\url{https://www.first.org}}, l'organisme qui coordonne l'action des différents CERTs et CSIRTs au niveau mondial.\\

Celui-ci réalise diverses missions réparties entre les trois divisions suivantes :\\

\begin{itemize}[itemsep=1em]
    \item[•] \textbf{La division en charge de la supervision et de la détection des événements de sécurité numérique ou « Security Operations Center » (SOC-MC)}\\
    Cette division est dédiée à la supervision et à la détection des événements de sécurité numérique. Sa mission principale est de protéger les systèmes d'information de la Principauté de Monaco contre les cybermenaces en temps réel, en assurant une vigilance constante et une réponse rapide aux évènements de sécurité. Une fois identifiés et analysés, ces évènements peuvent être déclarés comme incidents de sécurité lorsque leur dangerosité est avérée.
    \item[•] \textbf{La division en charge de la réponse aux incidents de sécurité numérique ou « Computer Security Incident Response Team » (CSIRT-MC)}\\
    Cette division a pour mission principale de répondre aux incidents de sécurité numérique (une fois ces derniers déclarés par le SOC-MC) en fournissant une assistance technique, des analyses approfondies et des solutions de remédiation aux parties prenantes (Étatiques et OIV publics ou privés) victimes.
    \item[•] \textbf{La division en charge de l’analyse, du partage et de l’information ou 
    « Information Sharing and Analysis Center » (ISAC-MC)}\\
    Cette division réalise la collecte, l'analyse et le partage d'informations liées à la cybersécurité provenant de diverses sources. Elle agit comme un centre névralgique pour la coordination des efforts de sécurité numérique, favorisant la collaboration entre les différents acteurs nationaux et internationaux.\\
\end{itemize}

Dans le cadre de mon stage, j'ai travaillé de façon concomitante au CSIRT-MC et au SOC-MC, sous la supervision de Bruno VALENTIN (mon tuteur de stage et responsable du CERT-MC) et de Sébastien ABBONDANZA (responsable du SOC-MC), car l'objectif de ma mission était d'améliorer certains processus de travail communs aux deux divisions. La suite de ce document se concentrera sur le CERT-MC; c'est auprès de ses équipes et avec son soutien que mon travail a été réalisé.