%%%%%%%%%%%%%%%%%%%%%%%%%%%%
% SECTION                  %
%%%%%%%%%%%%%%%%%%%%%%%%%%%%
\vspace{1em}

Mon expérience au sein de l'Agence Monégasque de Sécurité Numérique fut très enrichissante, tant sur le plan technique que professionnel. Ce stage m'a introduit au monde de la cybersécurité Étatique. J'ai pu côtoyer pour la première fois un CERT national et être témoin de leurs missions et organisations au quotidien. Ou encore travailler sur des problématiques concrètes, développer des solutions adaptées à des enjeux réels, et contribuer de manière significative à l'amélioration des processus de l'Agence.\\

Le projet de développement d'un outil automatisé de génération de règles de détection m'a permis de m'immerger dans des technologies clés des IDS de Suricata et de MISP, tout en approfondissant mes connaissances en programmation, notamment avec Python. La résolution de problèmes complexes, comme la gestion de milliers de règles et l'optimisation de leur efficacité, a développé ma capacité à analyser des situations et à proposer des solutions techniques robustes.\\

Sur le plan personnel, cette expérience m'a enseigné à mieux structurer mon travail, à collaborer avec des équipes multidisciplinaires, et à gérer un projet de bout en bout, depuis l'analyse des besoins jusqu'à la mise en production. J'ai également beaucoup appris en termes de communication, notamment en partageant régulièrement mes avancées avec mon tuteur de stage et les analystes du SOC-MC.\\

Enfin, cette période a renforcé mon intérêt pour le domaine de la cybersécurité et m'a confirmé dans mon choix de carrière. J'ai non seulement acquis des compétences techniques et méthodologiques, mais aussi développé une meilleure compréhension des enjeux stratégiques liés à la protection des systèmes d'information. Mon passage à l'AMSN m'a donné confiance en mes capacités à mener des projets ambitieux et à contribuer activement à la sécurisation des infrastructures numériques.