\markboth{\MakeUppercase{Glossaire}}{}%
\addcontentsline{toc}{chapter}{Glossaire}%

\begin{table}[ht!]
    \centering
    \small
    \begin{tabularx}{\textwidth}{|X|X|}
        \hline
        \textbf{AMSN (Agence Monégasque de Sécurité Numérique)} & Autorité nationale de l'Etat de Monaco en charge de la sécurité des systèmes d’information et des infrastructures critiques nationales  \\
        \hline
        \textbf{ANSSI (Agence Nationale de la Sécurité des Systèmes d'Information)} & Organisme français chargé de la protection des systèmes d'information de l'État et des infrastructures critiques nationales \\
        \hline
        \textbf{API (Application Programming Interface)} & Ensemble de règles et de protocoles permettant à des applications ou services de communiquer entre eux \\
        \hline
        \textbf{CERT (Computer Emergency Response Team)} & Groupe de spécialistes formés et expérimentés dans le domaine de la réponse aux incidents de sécurité informatiques \\
        \hline
        \textbf{CIRCL (Computer Incident Response Center Luxembourg)} & Centre national luxembourgeois de réponse aux incidents de sécurité informatique \\
        \hline
        \textbf{CSV (Comma-Separated Values) } & Format de fichier texte où les données sont organisées en lignes et colonnes, chaque valeur étant séparée par une virgule ou un autre délimiteur \\
        \hline
        \textbf{CTI (Cyber Threat Intelligence)} & Collecte et analyse de données relatives aux cybermenaces pour anticiper, prévenir et répondre aux attaques \\
        \hline
        \textbf{IDS (Intrusion Detection System)} & Système de sécurité qui surveille le trafic réseau ou les activités système pour détecter des comportements suspects ou des intrusions \\
        \hline
        \textbf{IOC (Indicator of Compromise)} & Signe observable, tel qu'une adresse IP malveillante ou un fichier suspect, qui indique une potentielle compromission ou intrusion dans un système \\
        \hline
        \textbf{MISP (Malware Information Sharing Platform)} & Plateforme open-source dédiée au partage d'indicateurs de compromission et de renseignements sur les cybermenaces \\
        \hline
        \textbf{OIV (Opérateur d'Importance Vitale)} & Organisation identifiée par l'État comme ayant des activités indispensables à la survie de la nation ou dangereuses pour la population \\
        \hline
        \textbf{SOC (Security Operation Center)} & Équipe centralisée chargée de surveiller, détecter et répondre aux menaces de sécurité en temps réel au sein d'une organisation \\
        \hline
    \end{tabularx}
\end{table}

\newpage

\begin{table}[ht!]
    \centering
    \small
    \begin{tabularx}{\textwidth}{|X|X|}
        \hline
        \textbf{VM (Virtual Machine)} & Environnement informatique simulé qui émule un système physique complet, permettant d'exécuter un système d'exploitation et des applications de manière isolée \\
        \hline
    \end{tabularx}
\end{table}