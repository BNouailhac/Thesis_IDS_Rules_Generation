%%%%%%%%%%%%%%%%%%%%%%%%%%%%
% SECTION                  %
%%%%%%%%%%%%%%%%%%%%%%%%%%%%
\vspace{0.75em}

Les cyberattaques représentent l'une des menaces les plus sérieuses pour les institutions publiques et les entreprises privées. Les attaquants, qu'il s'agisse de pirates informatiques ayant des objectifs malveillants ou d'agents étatiques, utilisent des méthodes de plus en plus sophistiquées pour s'infiltrer dans les systèmes, voler des données sensibles, perturber les opérations et causer des dommages financiers et/ou des atteintes à la réputation. \hyperref[biblio]{[8]}\\

Ces acteurs malveillants poursuivent l’amélioration constante de leurs capacités à des fins de gain financier, d’espionnage ou encore de déstabilisation. Cette amélioration s’illustre en particulier dans le ciblage des  attaquants qui cherchent à obtenir des accès discrets et pérennes aux réseaux de leurs victimes. Les acteurs malveillants tentent de compromettre des équipements périphériques qui  leur offrent un accès plus furtif et persistant. Ce ciblage périphérique se transpose également dans le type d’entités attaquées et confirme l’intérêt des attaquants pour les prestataires, les fournisseurs, les sous-traitants, les organismes de tutelle et l’écosystème plus large de leurs cibles finales.\\

Cette amélioration continue des stratégies et des compétences des attaquants met en évidence les limites de la sécurité des réseaux. La mise en place d'architectures sécurisées (pare-feu, antivirus, réseaux locaux virtuels, etc.) ne garantit pas qu'une intrusion informatique soit impossible. Face à des attaquants suffisamment motivés, une erreur humaine ou une vulnérabilité encore inconnue (dite Zero-Day) finira par permettre une intrusion sur le réseau défendu ou sur celui d'un partenaire dont le réseau dépend, quelle que soit la qualité de la sécurité mise en place.\\

Cette faiblesse intrinsèque de la sécurité informatique rend obligatoire la mise en place, au sein des infrastructures numériques sécurisées, d'outils capables de gérer la possibilité que l'infrastructure défendue soit déjà compromise. C'est le rôle que jouent les systèmes de détection d'intrusion en permettant la détection des tentatives d'intrusion au sein d'un réseau supervisé. \hyperref[biblio]{[5]}