%%%%%%%%%%%%%%%%%%%%%%%%%%%%
% SECTION                  %
%%%%%%%%%%%%%%%%%%%%%%%%%%%%
\vspace{1em}

La détection des intrusions consiste à surveiller les événements qui se produisent dans un système ou un réseau informatique et à les analyser pour y déceler des signes d'intrusion, définis comme des tentatives pour compromettre la confidentialité, l'intégrité, la disponibilité ou pour contourner les mécanismes de sécurité d'un ordinateur ou d'un réseau.\\

Les intrusions sont causées par des attaquants qui accèdent aux systèmes depuis l'internet, par des utilisateurs autorisés des systèmes qui tentent d'obtenir des privilèges supplémentaires pour lesquels ils ne sont pas autorisés, et par des utilisateurs autorisés qui abusent des privilèges qui leur sont accordés. Les systèmes de détection d'intrusion (IDS) sont des produits logiciels ou matériels qui automatisent ce processus de surveillance et d'analyse.\\ 

Un IDS peut être configuré pour surveiller différents types de trafic, tels que les paquets de données, les connexions réseau, les logs de systèmes, etc. Lorsqu'il détecte une activité suspecte, l'IDS génère une alerte qui est envoyée, permettant de prendre des mesures pour bloquer l'attaque ou corriger la vulnérabilité. \hyperref[biblio]{[3]}\\

\vspace{1em}

\subsection{Types d'IDS }

\vspace{1em}

La façon la plus courante de classer les IDS est de les regrouper par source d'information. Certains IDS analysent les paquets du réseau, capturés à partir des flux passants au sein du réseau pour trouver les attaquants. D'autres IDS analysent les sources d'information générées par les systèmes d'exploitation ou les logiciels d'application présents sur les postes utilisateurs pour y trouver des signes d'intrusion. \hyperref[biblio]{[6 - 7]}

\vspace{1em}

\subsubsection{Les systèmes de détection d'intrusion réseau}

\vspace{0.5em}

La majorité des systèmes commerciaux de détection d'intrusion sont basés sur les réseaux. Ces systèmes de détection d'intrusion réseau (ou NIDS : Network Intrusion Detection System) détectent les attaques en capturant et en analysant les paquets du réseau. À l'écoute d'un segment de réseau ou d'un commutateur, un système de détection d'intrusion en réseau peut surveiller le trafic réseau affectant plusieurs hôtes connectés au segment de réseau.\\

\newpage

\begin{figure}[h]%
    \center%
    \includegraphics[width=0.9\textwidth]{assets/NIDS.png}
    \caption[Exemple de systèmes de détection d'intrusion réseau (NIDS) (source: \href{https://miro.medium.com/v2/resize:fit:4800/format:webp/1*Jbw1iMzBztCaUokJnqePBg.png}{techno-skills.com})]{Exemple de systèmes de détection d'intrusion réseau (NIDS)}\label{fig:NIDSexemple}
\end{figure}

\vspace{1em}

\textit{Avantages des IDS en réseau :}\\

\begin{itemize}[itemsep=1em]
    \item[•] Quelques IDS bien placés peuvent surveiller un grand réseau ;
    \item[•] Le déploiement d'IDS en réseau a peu d'impact sur un réseau existant. Les IDS basés sur le réseau sont généralement des dispositifs passifs qui écoutent sur un fil de réseaux sans interférer avec le fonctionnement normal d'un réseau ;
    \item[•] Les IDS en réseau peuvent être configurés de manière à être hautement sécurisés contre les attaques et très difficiles à détecter pour les attaquants.\\
\end{itemize}

\vspace{1em}

\textit{Inconvénients des IDS en réseau :}\\

\begin{itemize}[itemsep=1em]
    \item[•] Les IDS basés sur le réseau peuvent avoir des difficultés à traiter tous les paquets dans un réseau important ou très fréquenté et, par conséquent, ne pas reconnaître une attaque lancée pendant les périodes de fort trafic ;
    \item[•] Les IDS basés sur le réseau ne peuvent pas analyser les informations chiffrées. Ce problème s'aggrave car de plus en plus d'organisations (et d'attaquants) utilisent des réseaux privés virtuels ;
    \newpage
    \item[•] La plupart des IDS basés sur le réseau ne peuvent pas déterminer si une attaque a réussi ou non ; ils peuvent seulement discerner qu'une attaque a été initiée. Cela signifie qu'après la détection d'une attaque par un système IDS basé sur le réseau, un examen manuel (ou une corrélation automatique à l'aide d'autres sources de données comme des logs) devra être effectué afin de caractériser si l'attaque a été réussie.
\end{itemize}

\vspace{1em}

\subsubsection{Systèmes de détection d'intrusion au niveau de l'hôte}

\vspace{0.5em}

Les systèmes de détection d'intrusion au niveau de l'hôte (ou HIDS : Host-based Intrusion Detection System), reposant sur l'hôte, fonctionnent sur la base des informations collectées à l'intérieur des appareils des utilisateurs. Les IDS basés sur l'hôte utilisent normalement des sources d'information de deux types : les pistes d'audit du système d'exploitation et les journaux du système. Ils permettent également de prendre un instantané des fichiers système existants et de les comparer à l'instantané précédant pour générer des alertes en cas de modifications.\\

\vspace{1em}

\begin{figure}[h]%
    \center%
    \includegraphics[width=0.9\textwidth]{assets/HIDS.png}
    \caption[Exemple de systèmes de détection d'intrusion au niveau de l'hôte (HIDS) (source: \href{https://miro.medium.com/v2/resize:fit:4800/format:webp/1*Jbw1iMzBztCaUokJnqePBg.png}{techno-skills.com})]{Exemple de systèmes de détection d'intrusion au niveau de l'hôte (HIDS)}\label{fig:HIDSexemple}
\end{figure}

\vspace{1em}

\textit{Avantages des IDS basés sur les hôtes :}\\

\begin{itemize}[itemsep=1em]
    \item[•] Les IDS basés sur l'hôte, grâce à leur capacité à surveiller les événements locaux d'un hôte, peuvent détecter des attaques qui ne peuvent l'être par un IDS basé sur le réseau ;

\newpage

    \item[•] Les IDS basés sur l'hôte peuvent fonctionner dans un environnement où le trafic réseau est chiffré, les sources d'information étant générées avant le chiffrement des données et/ou après le déchiffrement des données sur l'hôte de destination ;
    \item[•] Les IDS basés sur l'hôte ne sont pas affectés par les réseaux commutés.\\
\end{itemize}

\vspace{1em}

\textit{Inconvénients des IDS basés sur les hôtes :}\\
\vspace{0.5em}
\begin{itemize}[itemsep=1em]
    \item[•] Les IDS basés sur l'hôte sont plus difficiles à gérer, car les informations doivent être configurées et gérées pour chaque hôte surveillé ;
    \item[•] Étant donné que les sources d'information des IDS basés sur l'hôte résident sur l'hôte ciblé par les attaques, l'IDS peut être attaqué et désactivé dans le cadre de l'attaque ;
    \item[•] Les IDS basés sur l'hôte ne sont pas bien adaptés à la détection des scans de réseau ou d'autres formes de surveillance ciblant l'ensemble d'un réseau, car l'IDS ne voit que les paquets de réseau reçus par son hôte.
\end{itemize}

\newpage

\subsection{Méthode de détection }

\vspace{1em}

Les technologies IDS utilisent de nombreuses méthodologies pour détecter les incidents. Ces méthodologies peuvent être regroupées en deux grandes catégories : celles basées sur les signatures et celles basées sur les anomalies.  La plupart des technologies IDS utilisent plusieurs méthodologies de détection, séparément ou intégrées, afin de fournir une détection plus large et plus précise. \hyperref[biblio]{[1]}

\subsubsection{Les IDS à base de signatures}
\label{chap2:IDSsignature}

\vspace{0.5em}

Les IDS à base de signatures ont une base de données comportant un ensemble de signatures d’attaques (base de signatures). Le principe de fonctionnement est le test de correspondance. Les données du réseau sont analysées et comparées aux signatures d’attaques connues stockées dans la base de signatures. En cas de correspondance, une alerte est émise. Un avantage de ce système est qu’il a une meilleure précision lorsque les règles de signature sont correctes. Cette base de signatures est en général pré-initialisée avec des données de l’éditeur de l’IDS et mise à jour régulièrement pour prendre en compte les nouvelles attaques.\\

La détection basée sur les signatures est très efficace pour détecter les menaces connues, mais largement inefficace pour détecter les menaces inconnues. Elle a l'avantage d'être facile à mettre en place, mais sa principale limite est que si l'attaquant est conscient des règles de détection en place, il peut modifier légèrement sa technique pour qu'elle ne soit plus détectable.\\

\begin{figure}[h]%
    \center%
    \includegraphics[width=0.9\textwidth]{assets/IDSbaseSignatures.png}
    \caption[Procédure de détection d’attaques d’un IDS à base de signatures (source: \href{https://techno-skills.com/wp-content/uploads/2020/12/image-4.png}{techno-skills.com})]{Procédure de détection d’attaques d’un IDS à base de signature}\label{fig:ids-signature}
\end{figure}

\newpage

\subsubsection{Les IDS à base d’anomalies}

\vspace{0.5em}

Cette catégorie d’IDS utilise un modèle statistique du fonctionnement de référence du réseau qui peut comprendre la bande passante utilisée, les protocoles définis pour le trafic, les ports et les périphériques qui font partie du réseau. Il surveille régulièrement le trafic réseau et le compare au modèle statistique. En cas d’anomalie ou de divergence, l’administrateur est alerté. Un avantage de ce système est qu’il peut détecter des attaques nouvelles dont le comportement s’éloigne suffisamment de la normale.\\

Malheureusement, les détecteurs d'anomalies et les systèmes de détection d'intrusion qui en découlent produisent souvent un grand nombre de fausses alertes, car les modèles normaux de comportement des utilisateurs et des systèmes peuvent varier considérablement sans être prévisibles. Ce type de détection est difficile à utiliser de manière efficace.\\

\begin{figure}[h]%
    \center%
    \includegraphics[width=0.9\textwidth]{assets/IDSbaseAnomalies.png}
    \caption[Procédure de détection d’attaques d’un IDS à base d’anomalies (source: \href{https://techno-skills.com/wp-content/uploads/2020/12/image-5.png}{techno-skills.com})]{Procédure de détection d’attaques d’un IDS à base d’anomalies}\label{fig:ids-anomalie}
\end{figure}

\subsubsection{\textit{Note}}
Dans le cadre de mes travaux, j'ai travaillé avec des systèmes de détection d'intrusion réseaux faisant de la détection par signatures. Ce choix technologique se justifie dans le contexte de l'AMSN par le besoin de posséder une capacité de détection pouvant fonctionner sur les réseaux de tous les OIV en même temps sans perturber leurs activités usuelles. L'utilisation de la détection à base d’anomalies et des Systèmes de détection d’intrusion au niveau de l’hôte serait beaucoup plus difficile par la diversité des OIV et demanderait un niveau de connaissance sur les réseaux supervisés que l'Agence n'a pas vocation à connaître.