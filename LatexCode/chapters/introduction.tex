\markboth{\MakeUppercase{Introduction}}{}%
\addcontentsline{toc}{chapter}{Introduction}%

%Welcome to \Ac{ITBS}. ~\\
%Again, welcome to \Ac{ITBS}. ~\\
%Your introduction goes here. ~\\
%Your introduction goes here. ~\\
Par la dépendance croissante de nos sociétés à l'égard des technologies numériques, la sécurité des systèmes d'information est devenue une priorité incontournable. Ces dernières années ont vu les cyberattaques se multiplier autant en nombre qu'en sophistication, menaçant la confidentialité, l'intégrité et la disponibilité des données des individus, des entreprises comme des États. \\

Pour faire face à ces défis croissants, la stratégie de Défense qui s'est la plus démocratisée est celle de la prévention par la détection préalable des menaces. L'une des technologies les plus utilisées à cette fin est le système de détection d'intrusion (IDS), qui analyse les flux du réseau afin d'alerter les utilisateurs en cas d'activité malveillante présumée.\\

C'est pourquoi cette thèse s'intéresse au sujet des IDS, à leur fonctionnement et à l'optimisation de leur exploitation autour de la problématique : "Comment mettre en place un outil automatisé de génération
de règles de détection d’intrusion ?".\\ L'objectif de cette thèse est de proposer une méthodologie pour concevoir et mettre en place un outil capable de générer automatiquement des règles de détection d'intrusion. Cette automatisation vise non seulement à accélérer la mise à jour des règles de détection, mais aussi à améliorer leur précision et leur efficacité pour faciliter le travail des analystes.\\

Ce travail s'est déroulé dans le cadre d'un stage de six mois au sein de l'Agence Monégasque de Sécurité Numérique (AMSN), l'organisme national de cybersécurité de la Principauté de Monaco (équivalent de l'Agence Nationale de la Sécurité des Systèmes d'Information (ANSSI) française). L'AMSN travaille quotidiennement à la protection des infrastructures nationales et des Opérateurs d'Importance Vitale (OIV) grâce au SOC-MC qu'elle abrite, offrant ainsi un cadre idéal pour étudier le fonctionnement d'un IDS devant s'adapter à un grand nombre d'acteurs différents sur un territoire restreint comme celui de la Principauté.\\

\newpage

La structure de cette thèse sera la suivante : dans un premier temps, je présenterai l'AMSN ainsi que la problématique de la thèse en détail puis je passerai en revue les concepts fondamentaux et les technologies existantes en matière de détection d'intrusion. Ensuite, je détaillerai la conception de l'outil automatisé, en décrivant les choix méthodologiques et les algorithmes utilisés. Enfin, j'exposerai les résultats des tests et des évaluations effectués, ainsi que les perspectives d'amélioration et les futures orientations de recherche.


%Voici une référence à l'image de la Figure \ref{fig:test} page \pageref{fig:test} et une autre vers la partie \ref{chap:2} page \pageref{chap:2}.
%On peut citer un livre\, \cite{caillois1} et on précise les détails à la fin du rapport dans la partie références.
%Voici une note\,\footnote{Texte de bas de page} de bas de page\footnote{J'ai bien dit bas de page}. Nous pouvons également citer l'Algorithme , la Définition \ref{def1}, le Théorème \ref{theo1} ou l'Exemple \ref{exo1}...\\

%Le document est détaillé comme suit : le chapitre \ref{chap:chapterone} introduit le cadre général de ce travail. Il s'agit de présenter l'entreprise d'accueil et de détailler la problématique. Le chapitre \ref{chap:2} introduit les données ainsi que les modèles choisies.