\markboth{\MakeUppercase{Conclusion et Perspectives}}{}%
\addcontentsline{toc}{chapter}{Conclusion et Perspectives}%

Le développement d’un outil de gestion centralisée de règles de détection d’intrusion représente une étape clé dans l'amélioration de la réactivité et de la pertinence des systèmes de défense contre les cybermenaces. Ce document et le travail qu'il présente a permis d’apporter des solutions à une problématique complexe, tout en répondant aux exigences opérationnelles d’un organisme en charge de la protection des infrastructures critiques.\\

L'automatisation de la gestion des règles de détection permet non seulement de simplifier et d'accélérer les processus de mise à jour, mais aussi d’améliorer la précision et l’adaptabilité des systèmes de détection d’intrusion dans un contexte où les menaces évoluent rapidement. Ce projet a démontré l’importance de combiner des approches techniques pointues avec une vision stratégique, afin de mieux répondre aux enjeux de cybersécurité auxquels sont confrontés autant les États que les organisations.\\

Cependant, ce travail ne représente qu’une première étape dans l’évolution vers des systèmes de détection plus intelligents et plus proactifs. Les perspectives d’amélioration sont nombreuses. D’une part, une meilleure collaboration entre les différentes entités publiques et privées, à travers le partage accru de renseignements sur les menaces, est un progrès souhaitable pour renforcer l’efficience des systèmes de détection d'intrusion. D’autre part, l'optimisation des règles en fonction des spécificités des infrastructures surveillées constitue un axe de développement essentiel pour minimiser les faux positifs et maximiser l’efficacité des sondes.\\

En conclusion, ce projet de fin d’études a non seulement contribué à répondre à des problématiques techniques concrètes, mais il ouvre également la voie à des évolutions futures dans le contexte étudié. Le cheminement entrepris dans ce cadre est le reflet d’un besoin constant d’innovation et d’adaptation face à un environnement en perpétuelle mutation, où la sécurité des systèmes d’information demeure un enjeu de premier plan.\\